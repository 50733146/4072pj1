\documentclass[14pt,a4paper]{article}  
%紙張設定
\usepackage{xeCJK} 
%中文字體模組
\setCJKmainfont{標楷體} 
%中文字體
\usepackage[margin=2.5cm]{geometry}
\usepackage{titlesec}
\usepackage{titletoc}
\usepackage{CJK}
\usepackage{CJKnumb}
%%%%%%%%%%%%%%%%%%%%%%%%%%%%%%%%%目錄設定%%%%%%%%%%%%%%%%%%%%%%%%%%%%%%%%%%
%\titleformat{command}[shape]{format}{label}{sep}{before-code}[after-code]
%上面参数对应的含义如下所示。
 %command:大纲级别命令,如\chapter等。
 %shape:章节的预定义样式,分为9种。
 	%hang.默认值,标题在右侧,紧跟在标签后。
 	%block.标题和标签封装排版,不允许额外的格式控制。
 	%display.标题另起一段,位于标签的下方。
 	%runin.标题与标签同行,且正文从标题右侧开始。
 	%leftmargin.标题与标签分段,位于左页边。
 	%rightmargin.类似leftmargin,位于右页边。
 	%drop.文字包围标题。
 	%wrap.类似drop,文本会自动调整以适应最长的一行。
 	%frame.类似display,但有框线。
 %format:用于设置标签和标题文字的字体样式,这里可以包含竖直空距,即标题文字到正文的距离。
 %label:用于设置标签的样式,比如“第\chinese\thechapter章”大概是ctexbook类的默认样式。
 %sep:标签和标题文字的水平间距,必须是LATEX的长度表达。当shape取display时,表示竖直空距;取frame时,表示标题到文本框的距离。
 %before:标题前的内容。
 %after:标题后的内容。对于hang,block,display,此内容取竖向;对于runin、leftmargin,此内容取横向;否则此内容被忽略。
\titleformat{\chapter}[display]
            {\Huge\bfseries}{\Large 第\CJKnumber{\thechapter}章}
            {1ex}{}[]
\titleformat{\section}[hang]
            {\Large\bfseries}{第\CJKnumber{\thesection}節}
            {1em}{}[]
\titleformat{\subsection}[hang]
            {\normalsize\bfseries}{第\CJKnumber{\thesubsection}小節}
            {1em}{}[]
\titlecontents{chapter}[0em]
              {}{\normalfont\normalsize\bfseries\makebox[4.1em][l]
              {第\CJKnumber{\thecontentslabel}章}}{}
              {\titlerule*[0.7pc]{.}\contentspage}
\titlecontents{section}[2em]
              {}{\normalfont\normalsize\makebox[5.1em][l]
              {第\CJKnumber{\thecontentslabel}節}}{}
              {\titlerule*[0.7pc]{.}\contentspage}
\titlecontents{subsection}[4em]{}
              {\normalfont\normalsize\makebox[6.1em][l]
              {第\CJKnumber{\thecontentslabel}小節}}{}
              {\titlerule*[0.7pc]{.}\contentspage}
%%%%%%%%%%%%%%%%%%%%%%%%%%%%%%%%%%%標題%%%%%%%%%%%%%%%%%%%%%%%%%%%%%%%%%%%              
\begin{document} %文件
\begin{titlepage}%開頭
\begin{center}   %標題  
\begin{huge}
\makebox[1.5\width][s]{國立虎尾科技大學}\\[18pt]
\makebox[1.5\width][s]{機械設計工程系}\\[18pt]
\makebox[1.5\width][s]{專題製作報告}\\[18pt]
%設定文字盒子 [方框寬度的1.5倍寬][對其方式為文字平均分分布於方框中]\\距離下方18pt
\end{huge}
\vspace{1.5cm}
%下移1.5cm
\Huge\textbf{強化學習在機電系統設計與控制中之應用}
%字體等級Huge\加粗
\vfill
%垂直向增加空白行
%%%%%%%%%%%%%%%%%%%%%%%%%%%%%%%%%%%%%人員%%%%%%%%%%%%%%%%%%%%%%%%%%%%%%%%%%

\begin{Large}%人員
\makebox[5cm][s]{指導教授:\quad 嚴\quad 家\quad 銘\quad 老\quad 師}\\[6pt]
\makebox[5cm][s]{班\qquad 級:\quad 四\quad 設\quad 三\quad 甲}\\[6pt]
\makebox[5cm][s]{學\qquad 生:\quad 李\quad 正\quad 揚\quad(40723110)}\\[6pt]
\makebox[5cm][s]{ \hspace{29.3mm}林\quad 于\quad 哲\quad(40723115)}\\[6pt]
\makebox[5cm][s]{ \hspace{29.3mm}黃\quad 奕\quad 慶\quad(40723138)}\\[6pt]
\makebox[5cm][s]{ \hspace{29.3mm}鄭\quad 博\quad 鴻\quad(40723148)}\\[6pt]
\makebox[5cm][s]{ \hspace{29.3mm}簡\quad 國\quad 龍\quad(40723150)}\\[6pt]
%設定文字盒子[寬度為5cm][對其方式為文字平均分分布於方框中]空白距離{29.3mm}\空白1em
\end{Large}
\end{center}
\end{titlepage}
\newpage
%%%%%%%%%%%%%%%%%%%%%%%%%%%%%%%%%%%%%%%%%%%%%%%%%%%%%%%%%%%%%%%%%%%%%%%%%%

\begin{center}

\LARGE\textbf{摘要}\\
\begin{LARGE}
\begin{flushleft}
\hspace{12pt} 產業中需要加速許多工法的演算,以達到最佳化,但不能以實體一直測試不同方法,成本與時間不允許,便可以利用許多感測器觀測數值,以類神經網路運算,在虛擬環境架設結構,遠端控制、更改數值。\\

 \hspace{12pt} 此專題是利用現成裝置冰球台,設置對應虛擬模擬環境,減少現實模擬參數設置、成本,再加入類神經網路之中的Policy gradient與Reinforcement Learning,訓練冰球達到對應最佳化。
\end{flushleft}
\begin{center}
關鍵字:Policy gradient、虛擬環境架設結構、Reinforcement Learning
\end{center}
\end{LARGE}
\newpage
\renewcommand{\contentsname}{\centerline{\fontsize{18pt}{\baselineskip}\selectfont\textbf{目\quad 錄}}}
\addcontentsline{toc}{section}{摘\quad 要}
\tableofcontents
\newpage
\renewcommand{\listfigurename}{\centerline{\fontsize{18pt}{\baselineskip}\selectfont\textbf{圖\quad 表\quad 目\quad 錄 }}}
\listoffigures
\newpage
\end{center}
\newpage
\section{前言}
\setcounter{page}{1}
\newpage
\begin{center}
\addcontentsline{toc}{section}{參考文獻 }
\LARGE\textbf 參考文獻\\
\end{center}
\begin{flushleft}
\begin{Large}
[1]\quad https://towardsdatascience.com/derivative-of-the-sigmoid-function-536880cf918e\\

[2]\quad https://towardsdatascience.com/adam-latest-trends-in-deep-learning-optimization-6be9a291375c\\
\end{Large}
\end{flushleft}

\end{document}
