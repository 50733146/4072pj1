\documentclass[12pt,a4paper]{article}
\usepackage[utf8]{inputenc}
%加這個就可以設定字體
\usepackage{fontspec}
%使用xeCJK,其他的還有CJK或是xCJK
\usepackage{xeCJK}
\usepackage{enumerate}
%設定英文字型,不設的話就會使用預設的字型
\usepackage{hyperref}
\usepackage{graphicx}
\usepackage{geometry}
\geometry{a4paper,scale=0.8}
\setmainfont{Times New Roman}
\usepackage{listings}
\usepackage{amsfonts} %數學簍空的英文字
\lstset{language=python}
%設定中英文的字型
%字型的設定可以使用系統內的字型,而不用像以前一樣另外安裝
\setCJKmainfont{標楷體}
%以下是新增的自定义格式更改
\usepackage[]{caption2} %新增调用的宏包
\renewcommand{\figurename}{fig} %重定义编号前缀词
%中文自動換行
\XeTeXlinebreaklocale "zh"

%文字的彈性間距
\XeTeXlinebreakskip = 0pt plus 1pt

%設定段落之間的距離
\setlength{\parskip}{0.3cm}
\title{Score Function}
\author{虎尾科技大學\\40723115\\ 林于哲}
\date{January 16 2021}

\begin{document}
\maketitle
\tableofcontents

\section{Log Derivative Trick}
機器學習涉及操縱機率。這個機率通常包含normalised-probabilities或 log-probabilities。能加強解決現代機器學習問題的關鍵點,是能夠巧妙的在這兩種型式間交替使用,而對數導數技巧就能夠幫助我們做到這點,也就是運用對數導數的性質。\\
\section{Score Functions}
對數導數技巧的應用規則是基於參數$\theta$梯度的對數函數$p(x:\theta)$,如下:\\
$$\nabla_\theta logp(x:\theta)=\frac{\nabla_\theta p(x:\theta)}{p(x:\theta)}$$\\
$p(x:\theta)$是likelihood ; function參數$\theta$的函數,它提供隨機變量x的概率。在此特例中,$\nabla_\theta logp(x:\theta)$被稱為Score Function,而上述方程式右邊為score ratio(得分比)。\\[6pt]

\begin{mini}{The score function has a number of useful properties:}\end{mini}
\begin{itemize}
\item The central computation for maximum likelihood estimation. Maximum likelihood is one of the dominant learning principles used in machine learning, used in generalised linear regression, deep learning, kernel machines, dimensionality reduction, and tensor decompositions, amongst many others, and the score appears in all these problems. 
\end{itemize}
\begin{itemize}
\item The expected value of the score is zero. Our first use of the log-derivative trick will be to show this.\\
$$\mathbb{E}_{p(x; \theta)}[\nabla_\theta \log p(\mathbf{x}; \theta)] =\mathbb{E}_{p(x; \theta)}\left[\frac{\nabla_\theta p(\mathbf {x}; \theta)}{p(\mathbf{x}; \theta)} \right]$$
$$= \int p(\mathbf {x}; \theta) \frac{\nabla_\theta p(\mathbf {x}; \theta)}{p(\mathbf{x}; \theta)} dx= \nabla_\theta \int p(\mathbf{x}; \theta) dx=\nabla_\theta 1 = 0$$\\
In the first line we applied the log derivative trick and in the second line we exchanged the order of differentiation and integration. This identity is the type of probabilistic flexibility we seek: it allows us to subtract any term from the score that has zero expectation, and this modification will leave the expected score unaffected (see control variates later).
\end{itemize}
\begin{itemize}
\item The variance of the score is the Fisher information and is used to determine the Cramer-Rao lower bound.\\
$$\mathbb{V}[\nabla_\theta \log p(\mathbf{x}; \theta)] = \mathcal{I}(\theta) =\mathbb{E}_{p(x; \theta)}[\nabla_\theta \log p(\mathbf{x}; \theta)\nabla_\theta \log p(\mathbf{x}; \theta)^\top]$$\\
We can now leap in a single bound from gradients of a log-probability to gradients of a probability, and back. But the villain of today's post is the troublesome expectation-gradient of Trick 4, re-emerged. We can use our new-found power—the score function—to develop yet another clever estimator for this class of problems.
\end{itemize}
\section{Score Function Estimators}





\end{document}  
