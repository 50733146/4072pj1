\begin{center}
\fontsize{11pt}{16pt}\sectionef\hspace{12pt} 摘要
\end{center}
\begin{flushleft}
%\fontsize{10pt}{16pt}\sectionef\hspace{12pt} 
產業中的需求提升因而生產、製造、研發等效率需要有所提升,最佳化的方式能有的提升效率。本專題將以實體冰球機的機電系統簡化進行強化學習,並結合類神經網路,測試不同演算法使冰球機對打控制系統最佳化,並嘗試將其算法應用到實體機電系統上。\\[12pt]

%\fontsize{10pt}{16pt}\sectionef\hspace{12pt} 
此專題是運用實體冰球對打機,將其導入CoppeliaSim模擬環境並給予對應設置,將其機電系統簡化並運用Open AI Gym進行訓練,找到適合此系統的演算法後,再到CoppeliaSim模擬環境中進行測試演算法在實際運用上的可行性。並嘗試透過架設伺服器將CoppeliaSim影像串流到網頁供使用者觀看或操控。\\[12pt]

\end{flushleft}
\begin{center}
\fontsize{10pt}{20pt}\selectfont 關鍵字: 類神經網路、強化學習、\sectionef CoppeliaSim、OpenAI Gym
\end{center}