\begin{flushleft}
{\large \timesbd{Methodology}}\\
\end{flushleft}

%方法論述
強化學習是環境和agent互動並互相影響著,以獎勵的方式鼓勵學習,並且不需要特別教導。類神經網路則是透過神經元之間的非線性與權重產生記憶性並學習,以back-propagation的方式修正權重與偏差達到進步的效果。結合類神經網路和強化學習這兩種算法,使訓練不需特別教導並以獎懲的方式鼓勵最大化得分,並在固定次數訓練後back-propagation修正權重偏差。\\

參考資料\\

解釋研究問題\\

描述研究框架\\
將對打系統的基本移動控制藉由強化學習來達到最佳化,透過Gym測試演算法可行性再進一步運用CoppeilaSim進行3D模擬環境訓練對打系統,後續可藉由冰球機結合網路攝影機將整個對打系統實際運用。

應用的方法\\

研究問題與理論和實踐相關\\

為什麼選擇的方法解決該問題\\

%藉由實體冰球機的零件透過Solidworks繪製、組立再導入虛擬環境