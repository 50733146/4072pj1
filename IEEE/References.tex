\begin{flushleft}
{\large \timesbd{References}}\\
\end{flushleft}
\begin{flushleft}\sectionef
[1]\hspace{1em}\href{https://gym.openai.com/}{\underline{https://gym.openai.com/}}\\[12pt]

[2]\hspace{1em} \href{https://www.coppeliarobotics.com/coppeliaSim}{\underline{https://www.coppeliarobotics.com/coppeliaSim}}\\[12pt]


[3]\hspace{1em} \href{https://www.coppeliarobotics.com/coppeliaSim}{\underline{https://www.coppeliarobotics.com/coppeliaSim}} \\[12pt]
\hspace{2.25em}強化學習是環境和agent互動並互相影響著,以獎勵的方式鼓勵學習,並且不需要特別教導。類神\\
\hspace{2.25em}經網路則是透過神經元之間的非線性與權重產生記憶性並學習,以back-propagation的方式修正權\\
\hspace{2.25em}重與偏差達到進步的效果。結合類神經網路和強化學習這兩種算法,使訓練不需特別教導並以獎懲\\
\hspace{2.25em}的方式鼓勵最大化得分,並在固定次數訓練後back-propagation修正權重偏差。\\[12pt]

\iffalse
\parbox[t][5pt][l]{25pt}{[3]}
\begin{minipage}[t]{15pt}
  \underline{https://gym.openai.com/}
\end{minipage}

\makebox[25pt][l]{[3]}\makebox[45pt][l]{\raisebox{-3ex}[5\height]{\underline{https://gym.openai.com/}https://gym.openai.com/https://gym.openai.com/https://gym.openai.com/https://gym.openai.com/https://gym.openai.com/https://gym.openai.com/}}

\makebox[25pt][l]{[3]}\makebox[45pt][l]{\raisebox{-3ex}[5\height]{\underline{https://gym.openai.com/}}}
\fi
\end{flushleft}
