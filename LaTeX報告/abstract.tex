\renewcommand{\baselinestretch}{1.5} %設定行距
\pagenumbering{roman} %設定頁數為羅馬數字
\clearpage  %設定頁數開始編譯
\sectionef
\addcontentsline{toc}{chapter}{摘~~~要} %將摘要加入目錄
\begin{center}
\LARGE\textbf{摘~~要}\\
\end{center}
\begin{flushleft}
\fontsize{14pt}{20pt}\sectionef\hspace{12pt} 產業中的需求提升因而生產、製造、研發等效率需要有所提升,最佳化的方式能有的提升效率。本專題將以實體冰球機的機電系統簡化進行強化學習,並結合類神經網路,測試不同演算法使冰球機對打控制系統最佳化,並嘗試將其算法應用到實體機電系統上。\\[12pt]

\fontsize{14pt}{20pt}\sectionef\hspace{12pt} 此專題是運用實體冰球對打機,將其導入CoppeliaSim模擬環境並給予對應設置,將其機電系統簡化並運用Open AI Gym進行訓練,找到適合此系統的演算法後,再到CoppeliaSim模擬環境中進行測試演算法在實際運用上的可行性。\\[12pt]
\end{flushleft}
\begin{center}
\fontsize{14pt}{20pt}\selectfont 關鍵字: 類神經網路、強化學習、\sectionef CoppeliaSim、OpenAI Gym
\end{center}
\newpage
%=--------------------Abstract----------------------=%
\renewcommand{\baselinestretch}{1.5} %設定行距
\addcontentsline{toc}{chapter}{Abstract} %將摘要加入目錄
\begin{center}
\LARGE\textbf\sectionef{Abstract}\\
\begin{flushleft}
\fontsize{14pt}{16pt}\sectionef\hspace{12pt} As the demand in the industry increases, the efficiency of production, manufacturing, research and development needs to be improved, and the optimization method can improve the efficiency. This topic will use the simplification of the electromechanical system of the physical ice hockey machine for reinforcement learning, combined with neural networks, test different algorithms to optimize the ice hockey machine sparring control system, and try to apply its algorithm to the physical electromechanical system.\\[12pt]

\fontsize{14pt}{16pt}\sectionef\hspace{12pt} This project is to use the physical air hockey to play machine, introduce it into the CoppeliaSim simulation environment and give the corresponding settings, simplify its electromechanical system and use Open AI Gym for training, find an algorithm suitable for this system, and then perform it in the CoppeliaSim simulation environment Feasibility of testing algorithm in practical application.\\
\end{flushleft}
\begin{center}
\fontsize{14pt}{16pt}\selectfont\sectionef Keyword:  nerual network、reinforcement learning、 CoppeliaSim、OpenAI Gym
\end{center}